
%% bare_conf.tex
%% V1.4b
%% 2015/08/26
%% by Michael Shell
%% See:
%% http://www.michaelshell.org/
%% for current contact information.
%%
%% This is a skeleton file demonstrating the use of IEEEtran.cls
%% (requires IEEEtran.cls version 1.8b or later) with an IEEE
%% conference paper.
%%
%% Support sites:
%% http://www.michaelshell.org/tex/ieeetran/
%% http://www.ctan.org/pkg/ieeetran
%% and
%% http://www.ieee.org/

%%*************************************************************************
%% Legal Notice:
%% This code is offered as-is without any warranty either expressed or
%% implied; without even the implied warranty of MERCHANTABILITY or
%% FITNESS FOR A PARTICULAR PURPOSE! 
%% User assumes all risk.
%% In no event shall the IEEE or any contributor to this code be liable for
%% any damages or losses, including, but not limited to, incidental,
%% consequential, or any other damages, resulting from the use or misuse
%% of any information contained here.
%%
%% All comments are the opinions of their respective authors and are not
%% necessarily endorsed by the IEEE.
%%
%% This work is distributed under the LaTeX Project Public License (LPPL)
%% ( http://www.latex-project.org/ ) version 1.3, and may be freely used,
%% distributed and modified. A copy of the LPPL, version 1.3, is included
%% in the base LaTeX documentation of all distributions of LaTeX released
%% 2003/12/01 or later.
%% Retain all contribution notices and credits.
%% ** Modified files should be clearly indicated as such, including  **
%% ** renaming them and changing author support contact information. **
%%*************************************************************************


% *** Authors should verify (and, if needed, correct) their LaTeX system  ***
% *** with the testflow diagnostic prior to trusting their LaTeX platform ***
% *** with production work. The IEEE's font choices and paper sizes can   ***
% *** trigger bugs that do not appear when using other class files.       ***                          ***
% The testflow support page is at:
% http://www.michaelshell.org/tex/testflow/



\documentclass[conference]{IEEEtran}
\usepackage{listings}	

% *** GRAPHICS RELATED PACKAGES ***
%
\ifCLASSINFOpdf
  % \usepackage[pdftex]{graphicx}
  % declare the path(s) where your graphic files are
  % \graphicspath{{../pdf/}{../jpeg/}}
  % and their extensions so you won't have to specify these with
  % every instance of \includegraphics
  % \DeclareGraphicsExtensions{.pdf,.jpeg,.png}
\else
  % or other class option (dvipsone, dvipdf, if not using dvips). graphicx
  % will default to the driver specified in the system graphics.cfg if no
  % driver is specified.
  % \usepackage[dvips]{graphicx}
  % declare the path(s) where your graphic files are
  % \graphicspath{{../eps/}}
  % and their extensions so you won't have to specify these with
  % every instance of \includegraphics
  % \DeclareGraphicsExtensions{.eps}
\fi
% graphicx was written by David Carlisle and Sebastian Rahtz. It is
% required if you want graphics, photos, etc. graphicx.sty is already
% installed on most LaTeX systems. The latest version and documentation
% can be obtained at: 
% http://www.ctan.org/pkg/graphicx
% Another good source of documentation is "Using Imported Graphics in
% LaTeX2e" by Keith Reckdahl which can be found at:
% http://www.ctan.org/pkg/epslatex
%
% latex, and pdflatex in dvi mode, support graphics in encapsulated
% postscript (.eps) format. pdflatex in pdf mode supports graphics
% in .pdf, .jpeg, .png and .mps (metapost) formats. Users should ensure
% that all non-photo figures use a vector format (.eps, .pdf, .mps) and
% not a bitmapped formats (.jpeg, .png). The IEEE frowns on bitmapped formats
% which can result in "jaggedy"/blurry rendering of lines and letters as
% well as large increases in file sizes.
%
% You can find documentation about the pdfTeX application at:
% http://www.tug.org/applications/pdftex




% correct bad hyphenation here


\begin{document}
\title{Critique of Configurations Everywhere: Implications for Testing and Debugging in Practice\\
for CS5371 Soft Test for Mobile \& Emb Sys}

\author{\IEEEauthorblockN{Jeremy Solmonson}
\IEEEauthorblockA{School of Security Engineering\\
University of Colorado at Colorado Springs\\
Colorado Springs, CO 80922\\
Email: jsolmons@uccs.edu}}

\maketitle

\begin{abstract}
This is paper is a critique of "Configurations Everywhere: Implications for Testing and Debugging in Practice" In accordance with homework 5 requirements the critique will be based on: suggestion for acceptance, summary of paper, evaluation of paper, positive points, negatives points, and potential future work. 
\end{abstract}

\IEEEpeerreviewmaketitle

\section{Suggestion for Acceptance}
I would recommend to accept this paper with minor revisions. The paper was well written and made some unique insights into software configurations. 

\section{Summary of Paper}
This paper attempts to examine issues with debugging and fixing applications that fail due to conflicting configuration settings. Due to the numerous settings in current applications the debugging process can be challenging. The output from a fault or error is usually insufficient to isolate and fix the issue by using only the error information. To assist, the researches examined the configuration space of three applications. They found that large program are usually built with multiple programming languages, there are many different ways to modify a software configuration setting, and the configuration state usually cannot be determined upon failure. Additionally, they noted four lessons learned to help overcome these observations. 

\section{Evaluation}
This paper was well organized and clearly articulated intricate concepts. The paper effectively discussed the experiment and tied together the results with lessons learned. The abstract could discuss more impact on why this research is more useful. I believe the intent is to help solve software errors and faulty systems. This was briefly discussed but not overtly stated within the abstract. While only three applications were analyzed, the level of detail and walk through analysis was easy to follow and readable. Overall I would give an exceptional rating to the readability of this paper.

While the readability was well done, the scope of the experiment was severely limited. The authors only assessed three applications, one of which was not disclosed in the paper. Although the results were displayed, we can assume the undisclosed application performed a similar function to Firefox or LibreOffice. If these three applications perform similar functions, then similar results can be expected from each application. As a result, the level of diversity was also limited in this experiment. 

While the lessons learned stated briefly stated, "what needs to be done" they didn't address "how to do this." Specifically, what actions can future researchers do to improve up this model. The lessons learned was mostly targeted toward software developers on managing their own projects. A lot of manual work  is required to perform the lessons learned outlined within this paper. The areas that researchers could focus toward provided minimal insight on how to achieve the desired result. 

\section{Positive Points}
+ High readability. This paper was well organized and articulated the details. 

+ The lessons learned targeted areas for researchers and practitioners on next steps to improve the configuration domain. 

\section{Negative Points}
- Limited experiment scope. Experiments should be conducted on a larger diverse set of software.

\section{Potential Future Work}
The lessons learned identified four areas that could show improvement within the configuration domain. Two were targeted towards practitioners and two were targeted toward researchers. The two that were targeted toward researchers were: create software to better identify getter and setter functions and capture the configurations when an application fails.

I think the authors are in the right direction for the desired research directions, the ability to that end state can be challenging. The next focus of research would be to identify areas of code based on a set of conditions. While the authors specifically states getter and setter functions, the ability to search through code that performs desired functions is valuable by itself. A program could be created to allow the author to input a set of unique conditions, and that code could be searched (dynamic, static, or both) to find sections of code that performs those functions. To the best of my knowledge, that does not exist. 

As for the capturing configurations portion of future research, I do not see this as possible for every unique application. I think the authors of individual software will have to consider that as the program is developed. The ability to provide a platform that fits for all programs would be nearly impossible. Java is a good example as the bytecode was supposed to be able to run on any machine. The issue is the overhead to use java, and when applications require more speed, developers choose other languages. 

As of this moment, I cannot think of any other direction to take this research that was not already mentioned. 
 

% references section
%\nocite{*}
%\bibliographystyle{IEEEtran}
%\bibliography{Week3}


%\end{thebibliography}


\end{document}


